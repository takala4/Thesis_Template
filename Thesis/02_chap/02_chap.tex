\documentclass[../main/main]{subfiles}
\setcounter{chapter}{1}

\begin{document}
\chapter{FAQ}
\section*{まずは...}
  \begin{itemize}
    \item コンパイルエラーの多くは,エラーメッセージをそのままググることで,解決策が見つかることが多い.まずは,エラーメッセージをググること.chatgptに聞くのも良い.
    \item 一時ファイルを削除してからコンパイルすると,エラーが解消されることがある.
  \end{itemize}

\section*{Q1. 参考文献を出力したいが,subfileごとに出力されてしまう}
main.texの\verb|\bibdummy{1}|を\verb|\bibdummy{0}|に変更する.

\section*{Q2. 日本語文献の引用がうまくいかない}
\ref{subsec:japanese_bib}節のように記述すること.

\section*{Q3. 章番号がずれている}
サブファイルの冒頭の\verb|\setcounter{chapter}{n}|を,章番号-1の値に変更する.


\section*{Q4. overleafでうまくコンパイルできない}
\cref{sec:1_overleaf}を参照のこと.ファイルの構成を変えたりした場合,中間生成ファイルを削除するとうまくいく場合がある.
\par
time outしてしまう場合は,refs.bibの位置をmainディレクトリに配置すると,main.texのコンパイルはできるようになる.
ただし,他のsubfileを個別にコンパイルする際は,文献の引用がうまくいかないの注意する必要がある.

\bibdummy{1}
\end{document}
